\documentclass[12pt]{article}

\usepackage[ansinew]{inputenc}
\usepackage[T1]{fontenc}
\usepackage{graphicx,textcomp,booktabs,amsmath,amsfonts}
\usepackage[scaled]{helvet}
\usepackage[ngerman]{babel}
\usepackage[a4paper, top=3.0cm, left=2.5cm, right=2.5cm, bottom=3.0cm]{geometry}
\usepackage[bf]{caption}
\usepackage{listings}
\usepackage[pdfborder={0 0 0}, colorlinks=false]{hyperref}

\renewcommand{\captionfont}{\footnotesize}
\renewcommand\floatpagefraction{0.9}
\renewcommand\topfraction{0.9}
\renewcommand\bottomfraction{0.9}
\renewcommand\textfraction{0.1}

\newenvironment{code}{\begin{tabbing}Links \= Mittl \= Mitte \= Mittr \= Rechts \kill}{\end{tabbing}}


\begin{document}
\pagestyle{empty}

\label{cover_page}
\pdfbookmark[0]{Cover}{cover_page}

\begin{center}
	\huge{GNU/Linux Befehle\\}
\end{center}

\begin{center}
	\today
\end{center}

\begin{abstract}
	\label{the_abstract}
	\pdfbookmark[0]{Abstract}{the_abstract}
	Dieses Dokument beschäftigt sich mit einigen GNU/Linux Befehlen die
	mir über den Weg gelaufen sind ;). Hoffentlich ist diese bei weitem nicht 
	komplette Liste von Befehlen auch noch für jemanden anderen nützlich.
	Bitte nicht wundern, es gibt Befehle zu den verschiedensten Themenbereichen. Wann immer
	ich lange mit einem Befehl gearbeitet habe, oder ein besonders tolles unix werkzeug mir
	die arbeit erleichter hat, wird es kurzerhand einfach hier niedergeschrieben...
\end{abstract}
\newpage

\pdfbookmark[0]{Contents}{Contents}
\tableofcontents

\newpage
\section{Grundlegendes}
Hier gibts die typischen Unix Befehle.
\subsection{Eine leere Datei fixer Gr"o"se erstellen}
Das erstellt eine 10 KB gro"se Datei.
\begin{code}
	\> \verb#cat /dev/zero | dd bs=1024 count=10 > ${datei}#
\end{code}
\subsection{Standard Ausgang in Datei(en) kopieren}
\begin{code}
	\> \verb#tee ${file1} ${file2} ...#
\end{code}
\subsection{Dateien auflisten die regulären Ausdruck beinhalten}
\begin{code}
	\> \verb#find ${directory} -exec grep -l -E "${regex}" {} \;#
\end{code}

\section{Mounten}
Hier die wichtigsten Befehle um Datein und Partitionen ins System einzuh"angen.
\subsection{Ein Iso-Image einh"angen}
Um iso-images zu mounten muss man folgenden Befehl ausf"uhren.
\begin{code}
	\> \verb#mount -o loop,ro ${iso_path} ${mountpoint}#
\end{code}
\subsection{Einen Ssh-Account einh"angen}
Ben"otigt wird:
\begin{itemize}
	\item[-] fuse
	\item[-] fuse-sshfs
\end{itemize}
Damit man als User fusermounts machen kann, muss man in der Gruppe \textit{fuse} 
sein (Benutzerverwaltung).\\
Um nun einen ssh-Account zu laden, muss man folgende Befehle ausf"uhren:
\begin{code}
	\> \verb#sshfs "${user}@${host}:${path}" ${mountpoint}#
\end{code}
Um den ssh Account wieder freizugeben muss man folgendes ausf"uhren:
\begin{code}
	\> \verb#fusermount -u ${mountpoint}#
\end{code}
\subsection{Ein Swapfile einh"angen}
Zuerst braucht man eine Datei mit der ben"otigten Gr"o"se (Dies kann z.B. mit dd erreicht werden).
Der folgende Befehl erstellt eine Swap-Partition in der Datei \textit{swapfile}:
\begin{code}
	\> \verb#mkswap ${swapfile}#
\end{code}
Nun wird dem System gesagt, dass es die Datei verwenden soll:
\begin{code}
	\> \verb#sudo swapon ${swapfile}#
\end{code}
Mit \textit{swapoff} kann man die datei danach wieder aush"angen.

\section{Die Packeteverwaltung}
\textit{Apt} ist eine Packeteverwaltung f"ur Ubuntu und sollte zum installieren benutzt werden.
Weiters kann man auch mit \textit{gdebi} Packete installieren und mit \textit{dpkg}.
\subsection{Programme installieren}
Dieser Code installiert das Programm mit Namen \textit{name}
\begin{code}
	\> \verb#sudo apt-get install ${name}#
\end{code}
Ein lokales Packet\footnote{eine *.deb Datei} installieren:
\begin{code}
	\> \verb#sudo gdebi ${packet}#
\end{code}

\subsection{Nur die Source-Codes herunterladen}
Dieser Code l"ad alle Datein f"ur das Packet \textit{name} herunter. Das Archive ist dann 
im gerade aktiven Verzeichnis zu finden.
\begin{code}
	\> \verb#apt-get source ${name}#
\end{code}

\subsection{Nur das *.deb Packet herunterladen}
Dieser Code l"ad alle Datein f"ur das Packet \textit{name} herunter. Das Archive ist dann unter 
\verb#/var/cache/apt/archives# zu finden.
\begin{code}
	\> \verb#sudo apt-get -d install ${name}#
\end{code}
\subsection{Andere Version von einem Program verwenden}
Um die zu verwendente Version\footnote{wurde bereits installiert} auszuw"ahlen
z.b. f"ur Java:
\begin{code}
	\> \verb#sudo update-alternatives --config java#
\end{code}
\subsection{Liste aller installierten Version}
Um eine Liste aller installierter Versionen zu bekommen kann folgender Befehl\footnote{Am Beispiel java}
 verwendet werden:
\begin{code}
	\> \verb#update-alternatives --list java#
\end{code}

\section{Netzwerk}
Netzwerkbefehle.
\subsection{Lokale Rechner}
Lokale Rechner werden unter \verb#${rechnername}.local# angesprochen.
\subsection{X-Server tunneln}
Das \textit{C} bei den Flags steht f"ur Kompression und sollte auch verwendet werden. Um das X11 Protokoll zu tunneln muss man das \verb#X# oder \verb#Y# Flag setzten
\begin{code}
	\> \verb#ssh -XC "${user}@${rechnerdomain}"# \\
	\> \verb#ssh -YC "${user}@${rechnerdomain}"#
\end{code}
\subsection{Netzwerk neustarten}
\begin{code}
	\> \verb#sudo /etc/init.d/networking restart#
\end{code}

\section{Programmieren}
Hier gehts um Programmiertools und alles was dazugeh"ort.
\subsection{Dfu-Programmer}
Brenner f"ur z.B. den \textit{at90usb162}. Man muss das Programm als root aufrufen (mit sudo) oder
chmod u+s setzten.
Um ein neues Programm in den At90 zu laden muss man ihn zuerst l"oschen und danach das Programm brennen:
\begin{code}
	\> \verb#(sudo) dfu-programmer at90usb162 erase# \\
	\> \verb#(sudo) dfu-programmer at90usb162 flash ${hexfile}#	
\end{code}
\subsection{Avrdude}
Tool um diverse ISP Brenner zu benutzen. Verwendet wird hier ein STK500v2 kompatibles Ger"at.
\begin{itemize}
	\item Avrdude Terminal Mode: Hier kann man diverse Befehle eingeben und interaktiv den avr auslesen:
	\begin{code}
		\> \verb#avrdude -p ${device} -c ${protocol} -P ${port} -t# \\
		\> \verb#avrdude -p m8 -c stk500v2 -P /dev/ttyACM0 -t#
	\end{code}
\end{itemize}

\section{Git Versionsverwaltungssystem}
Schon seit längerem benutze ich regelmäßig dieses äußerst Praktische Werkzeug.
Es gibt sehr gute Einführungen in dieses Thema, hier werde ich nur Befehle abdecken, die 
mir irgendwann nützlich waren und wo ich vielleicht schon x-mal nachschlagen musste...

\subsection{'diff' nachdem man 'add' aufgerufen hat}
Das kennen vielleicht viele: Grad hat man alle Dateien geadded die sich verändert habe und man
ist kurz davor die commit-message zu schreiben, da fällt einem nichtmehr ein was man eigentlich alles
geändert hat. Hier hilft:
\begin{code}
	\> \verb#git diff --cached#
\end{code}
am besten in einem neuen Terminal damit man das ganze ein wenig durchgehen kann und während man so
die groben Änderungen beschreibt...

\subsection{'merge' ohne die gesamte history}
Wenn man z.b. einen Test-Zweig hat und dort mehrere Commits macht, im Endeffekt aber 
dann nur einen endgültigen sinnvollen Stand hat möchte man vielleicht die Zwischenstände
nicht in den Hauptzweig übernehmen. Das kann man mit der \verb#--squash#
Option erreichen:
\begin{code}
	\> \verb#git merge --squash ${some_branch}#
\end{code}

\section{VirtualBox}
Nützliche Dinge über oder für VirtualBox.
\subsection{Virtualbox Kernelmodul installieren}
Man muss vor der Installation das Packet:
\begin{code}
  \> \verb#virtualbox-ose-modules-2.6.24-19-generic# 
\end{code}
... natürlich für die richtige Kernel Version installiert haben.

\subsection{COM Ports emulieren}
Um COM Ports zu emulieren kann die Einstellung 'Host-Pipe' (erzeuge Host Pipe angehackt)
gewählt werden. Als Dateiname kann einfach irgendein pfad zu einer Datei die möglichst nicht
existiert (da sie sonst wohl überschrieben wird) eingegeben werden.
Z.B.: nur com1 erzeugt bei mir eine Datei com1 im Home-Verzeichnis des gerade aktiven Benutzers.
Soweit ich das verstanden habe erzeugt das wohl einen unix socket. Sehr nützlich ist hier das Tool
\textit{socat} da es sehr viele Protokolle auf andere umsetzten kann. Mit folgendem Befehl
\begin{code}
	\> \verb#socat UNIX-CONNECT:${path_to_com_pipe} TCP-LISTEN:${port}# \\
	\> \verb#socat UNIX-CONNECT:${path_to_com_pipe} TCP-LISTEN:8040#
\end{code}
kann der unix-socket umgesetzt werden auf tcp. Danach kann man bequem mit
\begin{code}
	\> \verb#telnet localhost 8040 -e ^X#
\end{code}
darauf zugeriffen werden.

\section{Linux Specifisches}
Hier kommt Linux Specifisches herein (also alles, das wirklich den Kernel betrifft).

\subsection{Kernel Modul kompilieren}
Um kernel Module zu installieren, sollten die beiden Packete \textit{linux-headers} und \textit{build-essential} (auf Ubuntu basierenden Systemen) mittels folgender Zeile installieret werden:
\begin{code}
	\> \verb#sudo apt-get install linux-headers-`uname -r` build-essential#
\end{code}
Dies kann oft nötig sein, wenn zusätzliche Kernel Module wie zum Beispiel die Virtual Box Guest Additions installiert werden sollen.

\end{document}
